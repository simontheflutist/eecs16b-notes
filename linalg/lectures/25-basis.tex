\chapter{The Discrete Fourier Transform}
\begin{flushright}
  O beloved son of Aegeus, for the gods alone\\
  there is no growing old, no dying ever.\\
  Everything else all-powerful Time destroys.\\
  (Sophocles, \emph{Oedipus at Colonus}, tr.~Lefkowitz and Romm)
  % Earth's strength fials, the body's strength fails,\\
  % trust dies, distrust blooms in its place\\
  % and the same spirit of friendliness never stays\\
  % between one man and anoth
\end{flushright}
The DFT basis uses the cyclic properties of the \(N\)th roots of unity to decompose a finite, discrete signal into periodic frequency components.

\section{Roots of 1}
\subsection{There are \(N\) of them}
What complex numbers \(\omega\) satisfy the equation \(\omega^N = 1\)?
Obviously 1 works.
To find the others, let \(\omega = re^{j\theta}\).
\begin{align}
  \del{re^{j\theta}}^N &= 1\\
  r^N e^{jN\theta} &= 1
  \intertext{Equating magnitudes, \(\left|r^N\right| = \left|r\right|^N = 1\).
  As \(r\) is a nonnegative real number, this means \(r = 1\).
  It can be cancelled from the left side.}
  e^{jN\theta} &= 1
  \intertext{This is satisfied when \(N\theta = 0\). As the complex exponential is periodic with a period of \(2\pi\), }
  N\theta &= 0, 2\pi, 4\pi, \ldots \\
  \theta &= 0, \frac{2\pi}{N}, \frac{4\pi}{N} \ldots \\
  \intertext{This list of options runs out of unique angles when \(\theta\) reaches \(2\pi\) again.
  Therefore, pairing each of these angles with magnitude \(r=1\), our solutions for \(\omega\) are}
  \omega &= e^{\frac{2\pi}{N}nj}, \quad n = 0, \ldots, N - 1.
\end{align}
These \(N\) numbers are called the \(N\)th roots of unity.
They can be viewed as the first \(N\) powers of \(e^{\frac{2\pi}{N}j}\), which is a rotation \(\frac{1}{N}\) of the way around the complex Unit Circle.
We will use the notation \(\omega_N = e^{\frac{2\pi}{N}j}\),\footnote{Any power \(\omega_N^r\), where \(r\) is prime to \(N\), will also generate all \(N\) roots of unity! This fact is amazing! Please ask me why in office hours!}
so that the \(N\)th roots of unity are \(\omega_N^0, \omega_N^1, \ldots, \omega_N^{N-1}\).

\subsection{They add to 0}
Because the \(N\)th roots of unity are all powers of \(\omega_N\), when you multiply them by \(\omega_N\), they don't go anywhere.
They just trade places.
We can use this coincidence to show that the sum of all the \(N\)th roots of unity is 0.

Call this sum \(S\).
We will multiply it by \(\omega_N\) and find that nothing changes.
\begin{align}
  S
  &= \sum_{n=0}^{N - 1}
  \omega_N^n\\
  \omega_N S
  &= \sum_{n=0}^{N - 1}
  \omega_N^{n+1}
  \intertext{The last term of this summation, \(\omega_N^N\), equals 1. Reordering the terms,}
  &= \sum_{n=0}^{N - 1}
  \omega_N ^n = S
\end{align}
We have shown that \(S\) is a number that you can multiply by a nonzero number and get \(S\) back.
Therefore \(S = 0\).

\subsection{They come in conjugate pairs}
Every \(N\)th root of unity comes with its complex conjugate:
if \(\omega^N = 1\), then \(\del{\overline \omega}^N = 1\).
This comes from conjugating the characterstic equation:
\begin{align}
  \omega^N &= 1\\
  \overline{\omega^N} &= \overline 1\\
  \del{\overline \omega}^{\overline N} &=
  \del{\overline \omega}^{N} = 1.
\end{align}

\section{The DFT Basis}
Define a basis \(\left\{u_0, \ldots, u_{N-1}\right\}\) for \(\mathbb{C}^N\) as follows:\footnote{Indexing from \(0\) presents this basis in the customary order.}
\begin{align*}
  u_{k}[n] &= \frac{1}{\sqrt{N}} \omega_N^{kn},
\end{align*}
where \(u_k[n]\) denotes the \(n\)th coordinate of basis vector \(u_k\).%
\footnote{Many normalizations of this basis exist. We are using the \(\frac{1}{\sqrt{N}}\) normalization because it's orthonormal, but it is not the most common default in numerical computing packages.}

\begin{theorem}[DFT is orthonormal]
  The DFT basis, defined above, is orthonormal.
\end{theorem}
\begin{proof}
  We need to check that \(\innerprod{u_k}{u_{k'}}\) equals 1 when \(k = k'\) and 0 otherwise.
  \begin{align}
    \innerprod{u_k}{u_{k'}}
    &= \sum_{n = 0}^{N - 1} u_k[n] \conj{u_{k'}[n]}\\
    &= \frac{1}{N}
    \sum_{n = 0}^{N - 1} \omega_N^{kn} \omega_N^{-k'n}\\
    &= \frac{1}{N}
    \sum_{n = 0}^{N - 1} \omega_N^{(k - k')n}
    \intertext{If \(k = k'\) then the sum is \(N\) and \(\innerprod{u_k}{u_{k'}} = 1\). Otherwise, write this sum using \(\zeta = \omega_N^{k - k'}\).}
    &= \frac{1}{N} \sum_{n = 0}^{N - 1} \zeta^n
  \end{align}
  This sum is stable  under multiplication by \(\zeta\), as \(\zeta^N = 1 = \zeta^0\), so it equals 0.
\end{proof}

By convention, in DFT contexts, indices start at zero and are interpreted modulo \(N\) because \(N\) roots of unity are \(N\)-cyclic; for example, the negative index \(-i\) can have the same meaning as the positive index \(N - i\).

The change of coordinates \emph{from} the DFT basis is called \textbf{synthesis} and is represented by \(F^* \in \mathbb{C}^{n \times n}\).
\begin{align}
  F^*_{kn} &= \frac{1}{\sqrt{N}} \omega_N^{kn}\\
  F^* &= \frac{1}{\sqrt N}\begin{pmatrix}
    1 & 1 & \ldots & 1\\
    1 & \omega & \ldots & \omega^{N - 1} \\
    1 & \vdots & \ddots & \vdots \\
    1 & \omega^{N- 1} & \hdots & \omega^{\del{N - 1}\del{N - 1}}
\end{pmatrix} & \text{(\textbf{synthesis matrix})}\\
x &= F^* X & \text{(\textbf{synthesis equation})}
\intertext{The change of coordinates \emph{to} the DFT basis is called \textbf{analysis} and is represented by \(F \in \mathbb{C}^{n \times n}\).}
F_{kn} &= \conj{F^*_{nk}} = \frac{1}{\sqrt{N}} \omega_N^{-kn}\\
F &= \frac{1}{\sqrt N}\begin{pmatrix}
  1 & 1 & \ldots & 1\\
  1 & \omega^{-1} & \ldots & \omega^{-\del{N - 1}} \\
  1 & \vdots & \ddots & \vdots \\
  1 & \omega^{-\del{N- 1}} & \hdots & \omega^{-\del{N - 1}\del{N - 1}}
\end{pmatrix}
& \text{(\textbf{analysis matrix})}\\
X &= F x & \text{(\textbf{analysis equation})}
\end{align}

``Synthesis'' and ``analysis'' are both words of Greek origin.
``Synthesis'' has the meaning of ``to put together,'' and ``analysis'' has the meaning of ``to take apart.''
\begin{itemize}
  \item If \(x\) records equally spaced samples of a signal, then DFT analysis is the process of shredding \(x\) into its frequency components \(X\).
  \item If \(X\) records the frequency components of \(x\), then DFT synthesis is the process of reassembling \(x\).
\end{itemize}
In the next section we will develop the interpretation of the DFT basis vectors as ``frequency components.''

\section{DFT of a sinusoid}
If the sinusoid \(\alpha\cos(\frac{2\pi k}{N}t + \varphi)\),
where \(k\) and \(N\) are integers,
is sampled \(N\) times in intervals of \(\Delta = 1\) starting at \(t = 0\),
the resulting sample vector is
\begin{align}
  x[n] &= \alpha\cos\del{\frac{2\pi k}{N}n + \varphi},
  \quad n = 0, \ldots, N - 1\\
  &= \frac{\alpha}{2}
  \del{e^{\del{\frac{2\pi k}{N}n + \varphi}j}
  + e^{-\del{\frac{2\pi k}{N}n + \varphi}j}}\\
  &= \frac{\alpha}{2}
  \del{e^{\frac{2\pi k}{N}nj} e^{\varphi j}
  + e^{-\frac{2\pi k}{N}nj} e^{-\varphi j}
  }\\
  &= \frac{\alpha}{2}
  \del{\del{e^{\frac{2\pi}{N}j}}^{kn} e^{\varphi j}
  + \del{e^{\frac{2\pi}{N}j}}^{-kn} e^{-\varphi j}
  }\\
  &= \frac{\alpha}{2}
  \del{\omega_N^{kn} e^{\varphi j}
  + \omega_N^{-kn} e^{-\varphi j}
  }\\
  &=
  \frac{\alpha}{2} e^{\varphi j} \omega_N^{kn}
  +
  \frac{\alpha}{2} e^{-\varphi j} \omega_N^{-kn}
  \\
  &=
  \frac{\alpha \sqrt N}{2} e^{\varphi j} u_k[n]
  +
  \frac{\alpha \sqrt N}{2} e^{-\varphi j} u_{N - k}[n]
  \\
  \intertext{In vector form,}
  x &=
  \frac{\alpha \sqrt N}{2} e^{\varphi j} u_k
  +
  \frac{\alpha \sqrt N}{2} e^{-\varphi j} u_{N - k}.
  \intertext{
  This expresses \(x\) as a superposition of \(u_k\) and \(u_{N -k}\).
  If \(k = 0\) modulo \(N\), then \(u_k = u_0 = u_{N - k}\) and \(x = \del{{\alpha \sqrt N} \cos{\varphi}} u_0\).
  If \(k \neq 0\) modulo \(N\), then  the Discrete Fourier Transform of \(x\) is the following:}
  X[n]
  &= \begin{cases}
    \frac{\alpha \sqrt N}{2} e^{\varphi j}, &n = k\\
    \frac{\alpha \sqrt N}{2} e^{-\varphi j}, &n = N - k\\
    0, & \text{else}
  \end{cases}
\end{align}
Note the following:
\begin{itemize}
  \item \(x\) is full of samples, but \(X\) is mostly zero.
  \item \(u_k\) and  \(u_{N - k}\) are conjugate, and so are their coefficients in \(x\).
  \item The DFT can extract a ``phasor representation'' of a sampled sinusoid.
  \item (The DFT can't tell the difference between \(k\) and \(k + N\).)
\end{itemize}
